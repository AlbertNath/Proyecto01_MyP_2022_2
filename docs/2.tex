{\textbf{Análisis del problema:}}
\begin{itemize}
  \item \textit{¿Qué es lo que queremos obtener?}:\\
        R: Deseamos obtener la información climatológica de los aeropuertos de origen y destino
        con base en información recabada de los boletos de vuelo.\\

  \item \textit{¿Cuáles son los datos con los que contamos? ¿Son suficientes?}:\\
        R: Contamos con un archivo tipo \textit{Comma Separated Value (CSV)} que contiene los
        códigos IATA, el valor de longitud y latitud de los aeropuertos tanto de orígen como de
        destino.\\
        Consideramos que estos datos son suficientes ya que al ejecutar un \textit{API request}
        a la página de servivcio \textit{OpenWeatherMap} con los datos que disponemos nos
        proporcionará la salida de datos que buscamos obtener.\\

  \item \textit{¿Qué hace que el resultado obtenido resuelva el problema?}:\\
        R: El programa que implementa el algoritmo diseñado proporcionará una tabla donde se
        muestre el clima de orígen y destino de acuerdo a los tickets proporcionados, con la
        información de
        interés sobre el clima.\\

  \item \textit{¿Qué operaciones o construcciones se deben obtener para llegar a la
        solución?}:
        \begin{enumerate}
          \item Procesar el archivo \textit{CSV} para extraer la información.

          \item Obtener la información adecuada de los aeropuertos como latitud, longitud y
                el código IATA en un formato más accesible.

          \item Solicitar a la API información acerca del clima con los datos que previamente
                procesamos del \textit{CSV}.

          \item Depurar la respuesta de la API extrayendo solo los datos de interés.

          \item Dar un formato legible y amigable como salida del programa de esta información.
        \end{enumerate}

\end{itemize}
