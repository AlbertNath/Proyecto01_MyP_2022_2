{\textbf{Pensando a futuro:}}\\

\textbf{Mantenimiento:}\\
A futuro sería necesario rectificar el desempeño del programa con un dataset más grande pues el
aeropuerto podría ampliar el tamaño de los tickets que se requieren procesar.\\

\textbf{Mejoras:}\\
Podemos agregar una interfaz gráfica que despliegue los datos en una ventana por cada par no
repetido de origen-destino. Además podríamos incluir un asistente de voz aprovechando las librerías
con las que cuenta Python para hacer la aplicación más accesible.\\

Se podría considerar el hacer el programa interactivo para el usuario del aeropuerto; con el
objetivo de lograr esto, se buscaría un despliegue en un servicio de nube como \textit{Amazon Web
  Services} o \textit{Azure} que soporte la carga de peticiones, lo que nos lleva a concurrencia,
para procesar múltiples solicitudes al mismo tiempo.\\

\textbf{Costo:}\\
El tiempo aproximado que se ha invertido en este proyecto fue de poco más de 24 horas; como el
valor de un programador en el mercado (con base en ofertas de LinkedIn), estimamos que el coste de
total hasta ahora sería de entre 717 USD a 956 USD. A esto hay que agregar el coste del
mantenimiento y mejoras al proyecto que surjan a largo plazo.
